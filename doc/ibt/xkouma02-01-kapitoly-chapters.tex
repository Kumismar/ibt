% Tento soubor nahraďte vlastním souborem s obsahem práce.
%=========================================================================
% Autoři: Michal Bidlo, Bohuslav Křena, Jaroslav Dytrych, Petr Veigend a Adam Herout 2019

% Pro kompilaci po částech (viz projekt.tex), nutno odkomentovat a upravit
%\documentclass[../projekt.tex]{subfiles}
%\begin{document}

\chapter{Úvod}\label{1_uvod}


\chapter{Abstrakt}\label{2_abstrakt}


\chapter{Základy teorie formálních jazyků}\label{3_teorie}

\section{Abeceda, řetězec a jazyk}
\section{Chomského hierarchie}
\section{Bezkontextová gramatika}

\begin{definition}\label{def_bkg}
    \emph{Bezkontextová gramatika} je čtveřice $G = (N, T, P, S)$, kde:
    \begin{itemize}
        \item $N$ je konečná množina neterminálních symbolů,
        \item $T$ je konečná množina terminálních symbolů,
        \item $P$ je konečná množina přepisovacích pravidel ve tvaru $A \rightarrow x$, $A \in N$ a $x \in (N \cup \Sigma)^*$,
        \item $S \in N$ je výchozí symbol gramatiky.
    \end{itemize}
\end{definition}
\todo{derivacni krok sem misto k~syntakticke analyze?}

\section{Konečný automat}
\begin{definition}\label{def_konecny_automat}
    
\end{definition}

\section{\todo{doplnit vice veci?}}



\chapter{Důležité pojmy syntaktické analýzy \todo{rozdelit na vice kapitol? srazit nektere pojmy do jedne sekce/rozdelit do vice? jake dalsi pojmy doplnit?}}\label{5_teorie_sa}

\section{Derivační krok}
Myšlenka \emph{derivačního kroku} je aplikovat pravidlo z~množiny pravidel bezkontextové gramatiky, čímž se část původního řetězce přepíše na novou.
\begin{definition}
    Nechť $G = (N, T, P, S)$ je BKG, $u, v \in (N \cup T)^*$ a $p = A \rightarrow x \in P$. \\
    Potom $uAv$ přímo derivuje $uxv$ za použití $p$ v~$G$, zapsáno $uAv \Rightarrow uxv [p]$, zjednodušeně $uAv \Rightarrow uxv$.\\
    Je možné říci, že $G$ provádí derivační krok z~$uAv$ do $uxv$.
\end{definition}

\subsection*{Sekvence derivačních kroků}
\todo{definici jsem nasel v~prezentacich ifj, nicmene to nevypada nejformalneji; bude stacit nebo radsi z~literatury?}

\section{Množiny potřebné k~sestrojení LL tabulky}
\todo{tady by mohlo byt Empty(x), First(x), Follow(x), Predict(x) + algoritmy pro jejich sestaveni?}

\section{LL tabulka}
\todo{algoritmus pro sestaveni tabulky?}

\section{Zásobníkový automat}\label{5_7_zasobnikovy_automat}
Zásobníkový automat je rozšíření konečného automatu, popsaného v~definici \ref{def_konecny_automat}, o~zásobník, matematicky přesněji o~zásobníkovou abecedu a~počáteční symbol na zásobníku.

\begin{definition}\label{def_zasobnikovy_automat}
    \emph{Zásobníkový automat} (ZA) je sedmice $M = (Q, \Sigma, \Gamma, R, s, S, F)$, kde:
    \begin{itemize}
        \item $Q$ je konečná množina stavů,
        \item $\Sigma$ je vstupní abeceda,
        \item $\Gamma$ je zásobníková abeceda,
        \item $R$ je konečná množina pravidel tvaru $Apa \rightarrow wq$, kde $A \in \Gamma$, $p,q \in Q$, $a \in \Sigma \cup \{\varepsilon\}$,  
        \item $s \in Q$ je počáteční stav, 
        \item $S \in \Gamma$ je počáteční symbol na zásobníku,
        \item $F \subseteq Q$ je množina koncových stavů.
    \end{itemize}
\end{definition}

\subsection*{Rozšířený zásobníkový automat}\label{5_7_x_rozsireny_ZA}
Původní zásobníkový automat lze rozšírit o~další chování. Například o~možnost čtení více symbolů ze zásobníku než původního jednoho, tedy při přechodech mezi stavy měnit celé řetězce na vrcholu zásobníku.

\begin{definition}
    \emph{Rozšířený zásobníkový automat} (RZA) je sedmice $M = (Q, \Sigma, \Gamma, R, s, S, F)$, kde:
    \begin{itemize}
        \item $Q, \Sigma, \Gamma, s, S, F$ jsou definovány stejně jako u~ZA,
        \item $R$ je konečná množina pravidel ve tvaru $vpa \rightarrow wq$, kde $v, w \in \Gamma^*$, $p, q \in Q$, $a \in \Sigma \cup \{\varepsilon\}$.
    \end{itemize}
\end{definition}

\section{Prediktivní syntaktická analýza}

\section{Precedenční tabulka}
\todo{algoritmus pro sestrojeni tabulky?}

\section{Precedenční syntaktická analýza}

\section{Abstraktní syntaktický strom}


\chapter{Cooperating distributed gramatický systém}\label{4_CDGS}

\todo{nejsem si jisty prekladem, zatim nechavam v~anglictine, ale vypada to divne}
Cooperating distributed gramatický systém (CDGS) stupně~\emph{n} je systém gramatik, které mezi sebou sdílejí množinu neterminálů i~terminálů a~startovací symbol.

\begin{definition}
\text{CDGS je n-tice}~$\Gamma = (N, T, S, P_i, \ldots ,P_n)$ pro $1 \leq i \leq n$, kde:
\begin{itemize}
    \item $N$, $T$, a $S$ jsou definovány stejně jako v~definici \ref{def_bkg},
    \item $P_i$ je konečná množina pravidel ve tvaru $A\rightarrow x$, kde $A$~i~$x$ jsou definovány stejně jako v~definici \ref{def_bkg}, nazývaná \emph{komponentou} systému,
    \item $i$-tá gramatika systému se zapisuje jako $G_i = (N,T,S,P_i)$
\end{itemize}   
Alternativní definice pro CDGS je $\Gamma = ((N, T, S, P_1), \ldots , (N, T, S, P_n))$.
\end{definition}
\todo{staci takovyto popis? mam se vyhnout vysvetlovani "vlastnimi slovy" nebo je naopak dobre, ze pred definici +- uvedu, o~co se jedna?}

\section{Derivační krok v~CDGS}\label{4_1_derivacni_krok}
Notace derivačního kroku v~CDGS je
\begin{center}
    $x_i \Rightarrow^{f} y$,
\end{center}
což znamená, že řetězec $x \in (N \cup T)^{*}$ derivuje řetězec $y \in (N \cup T)^{*}$ v~$i$-té komponentě za použití \emph{derivačního režimu} $f$.

\subsection*{Derivační režimy}\label{4_1_x_derivacni_rezimy}

Prvním a~nejpřirozenějším příkladem je režim $*$ \todo{Jak tento režim nazvat?}. V~tomto případě stačí, aby řetězec $y$ byl derivovatelný z~řetězce $x$ v~$i$-té komponentě, zapsáno $x\Rightarrow^{*}y$ v~$G_i = (N,T,P_i,S)$.

Podobným příkladem je režim \emph{ukončovací}, který spočívá v~nutné derivaci řetězce v~dané komponentě, dokud je to možné. Značí se písmenem \emph{t}. Jsou dvě nutné podmínky, aby \emph{y} bylo derivovatelné z~\emph{x} v~komponentě $G_i$ režimem \emph{t}.
\begin{enumerate}
    \item $x \Rightarrow^{*} y$ v~$G_i = (N,T,P_i,S)$\,--\,v~dané komponentě lze posloupností derivačních kroků získat řetězec $y$ z~řetězce $x$,
    \item $y \nRightarrow z$ pro všechna $z \in (N \cup T)^{*}$\,--\,není jiný další řetězec, který by z~$y$ šel odvodit.
\end{enumerate}

Další derivační režimy:
\begin{itemize}
    \item alespoň \emph{k}~derivací, tedy $x_i \Rightarrow^{\geq k} y$,
    \item nejvíce \emph{k}~derivací, tedy $x_i \Rightarrow^{\leq k} y$,
    \item právě \emph{k}~derivací, tedy $x_i \Rightarrow^{=k} y$,
\end{itemize}
kde $k \in \mathbb{N} \cup \{0\}$ a $i$ symbolizuje $i$-tou komponentu gramatického systému.

Derivační režimy mohou být reprezentovány jako množina, což pomůže definovat další pojmy v~následující podkapitole o~generovaných jazycích.
\begin{definition}\label{def_der_rezimy}
    Nechť $k\in \mathbb{N}$ a $*$, $t$ představují derivační režimy. \\
    Potom množina $D = \{*, t\} \cup \{\leq k, \geq k, =k\}$ reprezentuje derivační režimy použitelné v~CD gramatických systémech.
\end{definition}

\section{Jazyk generovaný CD gramatickým systémem}
Než bude definován samotný jazyk, je vhodné definovat pomocnou množinu, která reprezentuje \emph{možné derivace} z~řetězců.
\begin{definition}\label{def_mozne_derivace}
    Nechť $\Gamma = (N, T, S, P_i, \ldots, P_n)$. \\  
    Potom $F(G_j,u,f)=\{v:u_j\Rightarrow^{f}v\},$ $1 \leq j \leq n,$ $f\in D,$ $u\in (N \cup T)^{*}$ je množina všech řetězců $v$ derivovatelných z~$u$ v~$j$-té komponentě za použití derivačního režimu $f$.
\end{definition}
\begin{definition}\label{def_generovany_jazyk}
    Nechť $\Gamma = (N, T, S, P_i, \ldots, P_n)$. \\  
    Jazyk generovaný systémem $\Gamma$ za derivačního režimu $f$, $L_f(\Gamma) = \{ w \in T^*:$ existují $v_0, v_1,\ldots, v_m$ takové, že $v_k \in F(G_{j_{k}}, v_{k-1}, f), 1 \leq k \leq m,$ $1 \leq j_k \leq n, v_0 = S, v_m = w$ pro $m \geq 1\}$.  
\end{definition} 
\todo{jsou definice v~tomto formatu v~poradku?}

Výsledný řetězec $w$, který vznikl postupnou derivací startovacího symbolu $v_0$.
Měl několik mezikroků, které jsou reprezentovány řetězci $v_1, \ldots, v_{m-1}$.
Každý řetězec $v_k$, kde $1 \leq k \leq m$ byl zderivován z~řetězce $v_{k-1}$ v~komponentě $G_{j_{k}}$, kde $1 \leq j_k \leq n$ za derivačního režimu $f$.


\chapter{Představení jazyka Koubp \todo{opet si nejsem jisty nazvem + co vsechno vubec napsat?}}\label{6_jazyk_koubp}
Jazyk \emph{Koubp} je založený na jazyce IFJ22, který je podmnožinou jazyka PHP 8, jenž byl specifikován v~rámci zadání projektu do předmětu Formální jazyky a překladače v~akademickém roce 2022/2023.
\todo{idealne citovat zadani projektu?}

Některé aspekty jazyků jsou společné. 
Oba dva jazyky jsou strukturované, podporují definici proměnných a~funkcí.
Hlavní tělo programu se skládá z~prolínání sekvence příkazů a~definic funkcí, které se mohou vzájemně rekurzivně volat.
Neexistuje funkce \texttt{main()}, jak lze nalézt například u~jazyka C \cite{ISO-C-Standard}.
V~uživatelem definovaných funkcích může být větvení, iterace a~další běžné konstrukce.
Veškeré proměnné jsou lokální, i~v~rámci hlavního těla programu.
Soubory se zdrojovým kódem nelze slučovat a~vytvářet tak jediný modul, který by bylo možné zkompilovat.
\todo{doplnit veci, ktere nejsou spolecne (nezabihat do detailu, vse bude specifikovano v podkapitolach)}

\section{Gramatický systém definujíci syntax jazyka Koubp}

\subsection*{Indexace neterminálů a význam pro implementaci} 

\subsection*{Deadlock mezi neterminály statement a codeBlock}

\section{Deklarace a definice funkcí}

\section{Příkazy}

\subsection*{Přiřazení}

\subsection*{Větvení}

\subsection*{Cyklus while}

\subsection*{Cyklus for}

\section{Výrazy}

\subsection*{Operátory}

\subsection*{Priorita operátorů}

\subsection*{Volání funkcí}

\section{Vestavěné funkce}

\subsection*{Vstupně-výstupní funkce}

\subsection*{Funkce pro typovou konverzi}

\subsection*{Funkce pro prácí s~řetězci}

\chapter{Implementace syntaktického analyzátoru pro jazyk Koubp \todo{Urcite rozdelit na vice sekci, co pripadne z~projektu doplnit?}}\label{7_implementace}

\section{Hlavní myšlenky}


\section{Lexikální analýza a nástroj Flex}
\section{Syntaktická analýza}
\section{Abstraktní syntaktický strom}


\chapter{Závěr}